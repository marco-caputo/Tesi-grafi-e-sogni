\section{Analisi dei contesti semantici di un sognatore}\label{sec:analisi-dei-contesti-semantici-di-un-sognatore}

In questa sezione verrà esplorato un caso di studio, risultato del tentativo di usare il modello dei grafi multi-livello
allo scopo di individuare dei contesti di parole di valenza sintattica e semantica all'interno del mondo onirico
di una particolare sognatrice, i cui sogni sono stati prelevati dal dataset di sogni di DreamBank.net disponibile
online.
Daremo forma allo spazio delle parole di Emma a partire da un vasto corpus di racconti di sogni, e analizzeremo la
struttura del grafo multi-livello risultante.

\subsection{Pre-elaborazione con strumenti di NLP}\label{subec:pre-elaborazione-con-NLP}
Per poter procedere con l'analisi dei sogni di Emma affinché si potesse costruire un grafo delle parole
da cui poter cogliere aspetti semantici, è stato necessario effettuare una pre-elaborazione dei testi
attraverso gli strumenti tipici dell'elaborazione del linguaggio naturale.
L'NLP (in inglese \textit{Natural Language Processing}) è di fatti quella disciplina a metà tra intelligenza artificiale
e linguistica che si occupa di individuare i metodi di elaborazione e analisi di dati che si presentano sotto forma di
linguaggio naturale, ovvero di linguaggi usati dell'essere umano. \newline

La pre-elaborazione ha quindi previsto l'applicazione della seguente sequenza di fasi al corpus di 1218 sogni
originario:
\begin{enumerate}
    \item Pulizia del testo. \newline \noindent
          Questa fase ha incluso la rimozione degli spazi bianchi superflui, la standardizzazione della
          formattazione del testo e la correzione di errori ortografici o di incoerenze. È stata inoltre eseguita la
          rimozione della punteggiatura, snellendo ulteriormente i dati testuali. Tali procedure di pulizia sono
          essenziali per garantire la qualità e la coerenza dei dati in ingresso, migliorando così l'affidabilità
          dell'analisi.
    \item Tokenizzazione. \newline \noindent
          Questa fase ha comportato la suddivisione del testo in parole singole dette \textit{token},
          passaggio fondamentale nell'elaborazione del linguaggio naturale, creando la base per le fasi successive.
    \item Rimozione delle stop words. \newline \noindent
          Le \textit{stop words} sono parole comuni (come ``the'', ``is'', ``at'', ``which'' e ``on'') che tipicamente non hanno
          un ruolo significativo nei compiti di NLP, specialmente in quelli legati ad analisi semantiche.
          In questo caso, la loro rimozione può certamente aiutare a concentrare l'analisi sulle parole più ricche di
          contenuto nelle narrazioni dei sogni, riducendo significativamente il rumore nei dati e mettendo in evidenza
          i termini più salienti.
    \item Lemmatizzazione. \newline \noindent
          La lemmatizzazione è un processo che riduce le parole alla loro forma base, chiamata \textit{lemma}.
          Ad esempio, ``running'' verrebbe lemmatizzato in ``run'' e ``better'' in ``good''.
          Questo passaggio aiuta a ridurre le diverse declinazioni delle parole ai loro concetti di base, riducendo
          la dimensionalità dei dati testuali e rivelando, potenzialmente, schemi sottostanti in modo più chiaro.
    %\item Estrapolazione dei sostantivi. \newline \noindent
    %      Attraverso una fase di Part-of-Speech (POS) tagging, ciascuna parola viene etichettata di con la sua categoria
    %      grammaticale di appartenenza, come quella dei sostantivi, verbi, aggettivi, avverbi, ecc.
    %      A partire da queste categorizzazioni, tutte le parole che non sono state etichettate come sostantivi vengono
    %      filtrate. Lo scopo di questa fase è quello di concentrare l'analisi sulle parole che sono generalmente
    %      considerate le parti del testo più significative e ricche di contenuto, che meglio possono coadiuvare
    %      l'analisi semantica.
    \item Filtraggio delle parole meno frequenti. \newline \noindent
          Questa fase ha comportato la rimozione delle parole meno frequenti all'interno dell'intero corpus di sogni
          di Emma. Questo passaggio è stato eseguito per ridurre la complessità del testo e concentrare l'analisi sui
          termini più rilevanti e significativi, con l'obiettivo di individuare le tematiche principali trattate
          nei sogni e l'ordine con cui esse appaiono nei testi.
    \item Costruzione del grafo delle parole. \newline \noindent
          Infine, è stato costruito un grafo diretto basato sulle singole parole all'interno delle sequenze
          di lemmi associate ad ogni sogno. Tali parole, in quanto prese singolarmente rispetto alla sequenza
          di appartenenza, si definiscono \textit{unigrammi}. In questo grafo, quindi, ogni nodo rappresenta
          un unigramma univoco, con una proprietà ``peso'' che indica il numero di occorrenze di quell'unigramma
          nell'intero corpus di sogni.
          Gli archi orientati del grafo collegano gli unigrammi dei lemmi che appaiono seguitamente nella
          sequenza associata ad ogni sogno. Per questo motivo, ad ogni arco corrisponde un \textit{bigramma}, ovvero
          un accostamento ordinato di due parole.
          Il peso di ciascun arco rappresenta il numero di occorrenze di quello specifico accostamento ordinato nel testo.
          In questo modo, la struttura del grafo cattura non solo le relazioni sequenziali tra le parole nei sogni,
          ma anche la frequenza e la forza di queste parole e relazioni.
\end{enumerate}


\subsection{Riduzione della connettività del grafo delle parole}\label{subsec:elaborazione-del-grafo-delle-parole}

Sebbene il grafo delle parole risultante dalla pre-elaborazione dei sogni sia frutto di un processo di
semplificazione e riduzione della complessità del testo, la sua analisi diretta attraverso l'uso del
grafo multi-livello ne ha rivelato sin da subito la sua elevata connettività derivante dalla natura del
linguaggio naturale: parole e concetti possono essere strettamente legati (e quindi ``vicini'') ad un ampio numero di
altri concetti che a loro volta possono collegarsi direttamente ad altri concetti sparsi per il grafo.
Il risultato è uno spazio fortemente ``aggrovigliato'' che non favorisce la formazione di gruppi di parole distinti e
ben definiti e che, di conseguenza, non favorisce un processo di contrazione di qualità nella
gerarchia del grafo multi-livello.

Per rendere chiaro questo aspetto, si prenda come esempio il grafo multi-livello $M = (G, \langle f_{C_1}, f_{C_2}\rangle)$
rappresentato in figura~\ref{fig:les-miserables-graph}, il cui grafo $G$ rappresenta il grafo delle relazioni di
co-apparizione dei personaggi del romanzo \textit{Les Misérables} di Victor Hugo,
e le cui funzioni di contrazione $f_{C_1}$ e $f_{C_2}$ rappresentano rispettivamente le funzioni di contrazione
per cricche non reciproche e componenti fortemente connesse.
Appare evidente di come la struttura del grafo $G$ sia stata contratta con un elevato tasso di contrazione,
producendo una struttura notevolmente più semplice e facilmente interpretabile. Per via degli schemi di
contrazione scelti, appare evidente come la struttura originale del grafo abbia rivelato la sua ordinatezza ai livelli
superiori: in media i personaggi del romanzo possono apparire assieme ad una cerchia ristretta di altri personaggi,
ad eccezione di personaggi principali che risultano collegarsi a questi gruppi più o meno isolati di nodi.
Questo rispecchia in parte la natura dello spazio bidimensionale in cui i personaggi sono collocati: personaggi
appartenenti a luoghi distanti difficilmente appariranno insieme. I personaggi principali, in quanto seguiti
nella narrazione nel mentre che si spostano nello spazio, permettono di rompere questa dimensionalità, e risultano
essere collegati a nodi distanti tra loro. \newline

Per risolvere questo problema legato all'alta connettività del grafo delle parole, si è deciso di applicare
una tecnica di potatura del grafo, che consiste nell'identificazione di uno scheletro portante del grafo
basato sulle relazioni più significative tra le parole e dell'esclusione delle relazioni meno significative in
accordo a tale scheletro, allo scopo di ``appiattire'' lo spazio delle parole.

\nlparagraph{Classificazione degli archi}

L'algoritmo di potatura del grafo si basa sulla rilevanza delle relazioni tra le parole, ovvero degli archi del grafo.
Sebbene la rilevanza di un bigramma possa essere definita e valutata in diversi modi - come ad esempio il calcolo
dell'informazione mutua puntuale (PMI) tra le parole - in questo caso si è scelto di basare la rilevanza
direttamente sull'informazione fornita dai pesi degli archi e dei relativi nodi coinvolti.
In particolare, la misura di rilevanza di un arco è stata definita come la somma dei rapporti tra il peso dell'arco
e i singoli pesi dei nodi su cui l'arco è incidente.
In formule, siano $w(u)$, $w(v)$ e $w(u, v)$ il peso del nodo $u$ e $v$ e dell'arco $(u, v)$ rispettivamente,
la rilevanza dell'arco $(u, v)$ è definita come:
\begin{equation*}
    relevance(u, v) = \frac{w(u, v)}{w(u)} + \frac{w(u, v)}{w(v)}
\end{equation*}

La formula fornisce un criterio elementare per distinguere la frequenza di coppie di parole dalla loro importanza:
un arco con un peso elevato rispetto ai pesi dei nodi coinvolti è considerato più rilevante di un arco con un peso
elevato rispetto alla media dei pesi degli altri archi.
Ad esempio, un arco con peso 10 tra due nodi con pesi 100 e 200 è considerato meno rilevante di un arco con peso 5
tra due nodi con pesi 8 e 10. \newline

La scelta di questa misura di rilevanza è dovuta alla facilità con cui è possibile associarla ad un nuovo valore
positivo che rappresenti una distanza semantica tra le parole collegate.
Essendo tale valore di rilevanza sempre compreso nell'intervallo $(0, 2]$, è possibile definire un valore di costo
semantico dell'arco $(u, v)$ come $\frac{1}{relevance(u, v)}$, che è inversamente proporzionale alla
rilevanza dell'arco.
Gli archi possono, quindi, essere classificati in base a questo valore di distanza che d'ora in avanti chiameremo
\textit{costo} dell'arco.

\nlparagraph{Algoritmo per la riduzione della connettività}

Un algoritmo che sfrutti l'informazione data dal costo degli archi per ridurre la complessità e la connettività
di un grafo potrebbe agire costruendo in maniera incrementale un nuovo grafo a partire da quello fornito in input,
che ne rappresenti una versione semplificata.
L'algoritmo proposto nel seguente pseudocodice considera gli archi del grafo in input $G$ in ordine crescente di
costo al fine di valutarne l'inclusione nel nuovo grafo semplificato $H$, sulla base di un ulteriore parametro in input
che chiamiamo \textit{threshold}.

In particolare, ogni nodo del grafo originale $G$ sarà presente nel grafo semplificato $H$. Mentre per ogni arco $(u, v)$
di $G$ considerato nel ciclo for a riga 4, se nella versione non orientata di $H$, ottenuta rimuovendo l'orientamento
degli archi, esiste già un cammino tra i nodi $u$ e $v$ e il costo del cammino minimo calcolato sul costo degli archi
è maggiore al threshold, l'arco $(u, v)$ viene ignorato, altrimenti viene aggiunto al grafo semplificato $H$.
Si noti che nello pseudocodice, così come nella notazione tipica, un costo di cammino minimo pari a $\infty$ indica
che non esiste un cammino tra i nodi.

\begin{algorithm}[H] \floatname{algorithm}{Algoritmo}
    \caption{REDUCE-CONNECTIVITY($G$, $threshold$)}\label{alg:reduce-connectivity}
    \begin{algorithmic}[1]
        \State Sia $H = (W, F)$ un nuovo grafo orientato, con $W = G.V$ e $F = \emptyset$
        \State Sia $H_u = (W_u, F_u)$ un nuovo grafo non orientato, con $W_u = G.V$ e $F_u = \emptyset$
        \State Ordina gli archi in $G.E$ in ordine crescente di costo
        \For {$(u, v) \in G.E$}
            \State Sia $d$ il costo del cammino minimo tra $u$ e $v$ in $H_u$ calcolato sulla base del costo degli archi
            \If{$d == \infty$ \textbf{or} $d < threshold$}
                \State $F = F \cup \{(u, v)\}$
                \State $F_u = F_u \cup \{\{u, v\}\}$ \Comment{L'arco aggiunto ha lo stesso costo di $(u, v)$}
            \EndIf
        \EndFor
        \State \textbf{return} $H$
    \end{algorithmic}
\end{algorithm}

In questo modo, l'algoritmo costruisce arco dopo arco un grafo semplificato $H$ basato sui collegamenti più
rilevanti tra le parole.
Gli archi che vengono ignorati sono quelli che collegano parole considerate distanti secondo i collegamenti più
rilevanti stabiliti nello scheletro provvisorio del grafo $H$.
Il grado di tolleranza alla distanza tra le parole è regolato dal parametro \textit{threshold},
che può essere impostato per ottenere un grafo più o meno semplificato: più basso sarà il threshold, minore
sarà il grado di connettività del grafo prodotto.

\nlparagraph{Complessità}
Il costo dell'algoritmo REDUCE-CONNECTIVITY è principalmente dominato dalla fase di ordinamento degli archi
e del calcolo dei cammini minimi tra i nodi del grafo non orientato $H_u$.
Utilizzando algoritmi di ordinamento efficienti, il costo computazionale dell'ordinamento a riga 3 è $O(|E| \log |E|)$.
Il calcolo dei cammini minimi tra i nodi di $H_u$ a riga 5 può essere eseguito in tempo $O(|V| \log{|V|} + |E|))$
utilizzando l'algoritmo di Dijkstra con coda di priorità gestita da heap di Fibonacci.
Essendo che il ciclo for a riga 4 viene eseguito una volta per ogni arco in $|E|$, il costo totale dell'algoritmo
risulta essere:
\begin{equation*}
      O(|E| \log |E|) + O(|E| \cdot (|V| \log{|V|} + |E|)) \quad = \quad
      O(|E| (\log |E| + |V| \log{|V|} + |E|))
\end{equation*}

\subsection{Analisi del grafo multi-livello}\label{subsec:analisi-del-grafo-multi-livello}







Dopo la creazione dei grafi, abbiamo eseguito il rilevamento dei cicli e l'individuazione delle componenti fortemente
connesse. Il rilevamento dei cicli identifica i loop chiusi nel grafo, dove una sequenza di parole porta eventualmente
di nuovo alla parola iniziale. Questi cicli potrebbero rappresentare schemi o temi ricorrenti nelle narrazioni dei sogni,
potenzialmente indicando motivi o modelli di pensiero persistenti nei sogni di Emma. L'individuazione delle componenti
fortemente connesse identifica i sottografi all'interno dei quali ogni nodo è raggiungibile da ogni altro nodo.
Nel contesto dell'analisi dei sogni, queste componenti potrebbero rappresentare cluster di concetti o temi strettamente
interrelati che frequentemente co-occorrono e si interconnettono all'interno dei sogni. L'applicazione di queste
tecniche di analisi del grafo al grafo di unigrammi diretto e pesato potrebbe rivelare strutture complesse all'interno
delle narrazioni dei sogni, come temi ricorrenti, schemi di pensiero persistenti o cluster di concetti correlati.
I cicli rilevati potrebbero indicare elementi circolari o ripetitivi nei sogni, mentre le componenti fortemente connesse
potrebbero rappresentare unità tematiche coerenti o strutture narrative.