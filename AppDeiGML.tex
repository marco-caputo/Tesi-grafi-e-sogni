\chapter{Applicazioni dei Grafi Multilivello}

\paragraph{Possibili applicazioni della contrazione}
Nel contesto generale dell'analisi di grafi, la contrazione di nodi e archi pu\`o essere utile per:
\begin{itemize}
    \item Ridurre la complessit\`a dell'analisi strutturale di un grafo, sia che esso debba essere processato
    attraverso algoritmi costosi, sia che esso debba essere graficamente visualizzato, rendendolo pi\`u facilmente
    interpretabile ed evidenziando le caratteristiche strutturali di interesse.
    \item Studiare l'interrelazione di caratteristiche strutturali di un grafo, che rappresenti la navigabilit\`a
    di uno spazio basato su componenti fortemente connesse, cicli, cricche ecc.
    \item Stabilire il grado di connettivit\`a di un grafo, individuando la rilevanza, il numero e la dimensione
    delle sue contrazioni.
    Misurare il grado di complessit\`a dello spazio rappresentativo del grafo, in base al numero
    di nodi e archi presenti nelle contrazioni (grafi derivanti da specifici domini tendono ad avere un certo grado
    di complessit\`a legato ad un concetto spaziale).
\end{itemize}

Nel contesto dell'analisi dei testi e della loro rappresentazione tramite grafi, la contrazione di nodi e archi
pu\`o essere utile per:
\begin{itemize}
    \item Individuare contesti sintattici (e possibilmente semantici) di parole e frasi, evidenziando l'interrelazione
    e la distanza tra gruppi di parole e frasi.
    \item Valutare la somiglianza di singoli brevi testi, come il racconto dei sogni, individuando le
    macro-caratteristiche strutturali comuni e le differenze tra esse.
    \item Individuare pattern ricorrenti di parole e frasi su un corpus pi\`u ampio di testi.
    \item Stabilire il grado di connettivit\`a del grafo delle parole in base al numero di gruppi di parole e di nodi
    presenti nelle contrazioni, e confrontarlo con grafi multilivello che rappresentino un controllo.
\end{itemize}
