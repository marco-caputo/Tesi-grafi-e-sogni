\chapter*{Conclusioni e Sviluppi Futuri}\addcontentsline{toc}{chapter}{Conclusioni e Sviluppi Futuri}

% Valutare il peso degli archi ai fini della contrazione

In conclusione, appare evidente di come l'applicazione di queste
tecniche di analisi potrebbero rivelare strutture celate all'interno delle narrazioni dei sogni, come temi ricorrenti,
schemi di pensiero persistenti o cluster di concetti correlati.

Come evidenziato da questi risultati, questo approccio analitico multilivello potrebbe offrire una prospettiva
sull'importanza strutturale evolutiva dei nodi e sull'intensificazione delle loro connessioni attraverso i processi di
aggregazione.
Mostra la complessità dinamica e la connettività all'interno delle esperienze oniriche, fornendo una comprensione più
completa rispetto a un'analisi a livello singolo, che catturerebbe solo una rappresentazione statica della struttura
iniziale senza chiarire queste trasformazioni dinamiche.

I lavori futuri potrebbero coinvolgere l'analisi descritta del grafo multilivello su un dataset più ampio e già
classificato di narrazioni oniriche, per rivelare una correlazione tra le strutture del grafo multilivello e valutare
gli stati psicologici dei sognatori.

È stata poi eseguita l'identificazione delle entità nominate (NER) per individuare e classificare le entità nominate
all'interno del testo. Questo processo aiuta a riconoscere e categorizzare elementi come persone, luoghi,
organizzazioni e altre entità specifiche del dominio rilevanti per il contenuto dei sogni. Il NER può rivelare
temi importanti o elementi ricorrenti nei sogni di Emma, come luoghi frequenti o figure significative.
È stata inoltre condotta un'analisi della frequenza delle parole come parte dell'analisi preliminare.
Questo passaggio implica il conteggio delle occorrenze di ogni parola unica nel corpus, fornendo intuizioni sui
termini più comuni utilizzati nei sogni di Emma. Tale analisi può mettere in evidenza temi predominanti, emozioni o
oggetti che appaiono frequentemente nelle narrazioni dei sogni, aiutando a comprendere la struttura linguistica
di base delle narrazioni dei sogni.