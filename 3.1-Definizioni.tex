\section{Definizioni e propriet\`a di base}\label{sec:definizioni-e-proprieta-di-base}

In questa sezione saranno proposte definizioni e propriet\`a di base di una struttura dati per la rappresentazione
di gerarchie di grafi a pi\`u livelli, in cui sia possibile ottenere informazioni sull'intera struttura gerarchica in
relazione ai singoli grafi che la compongono, come attraverso l'espansione e contrazione di nodi.
In particolare, riprendendo dalla terminologia e dai concetti esistenti in letteratura, si proporranno definizioni
orginali di \textit{Grafi decontraibili} e di \textit{Grafi multi-livello}.

\subsection{Grafo decontraibile}

    A partire dal concetto noto di grafo quoziente e di contrazione, si vuole definire una particolare
    tipologia di grafi quoziente che, oltre a rappresentare le caratteristiche strutturali ad alto livello
    dei grafi da cui sono derivati, siano in grado di mantenere l'informazione originale di questi ultimi, ed in
    che modo essa sia legata alla sua rappresentazione astratta. \newline
    Nasce cos\`{\i} il concetto di grafo decontraibile, che intuitivamente pu\`o essere considerato come un grafo
    in cui i nodi sono riconducibili ad un grafo e gli archi ad un insieme di archi tra i nodi dei grafi
    associati ai nodi coinvolti. \newline

    \begin{definition}[Grafo Decontraibile]
        Un \textbf{grafo decontraibile} \`e una quadrupla $G = (V, E, dec_V, dec_E)$ dove:
        \begin{itemize}
            \item $V$ \`e un insieme di elementi detti \textbf{supernodi};
            \item $E \subseteq V \times V$ \`e un insieme di coppie ordinate di supernodi, dette \textbf{superarchi};
            \item $dec_V : V \rightarrow \mathcal{G}_D$ \`e una biiezione tale per cui $dec_V(v) = (\mathcal{V}_v,
                \mathcal{E}_v, dec_{\mathcal{V}_v}, dec_{\mathcal{E}_v})$ \`e un grafo decontraibile rappresentato
                dal supernodo $v$;
            \item $dec_E : E \rightarrow (\mathcal{V} \times \mathcal{V})$ con $\mathcal{V} = \bigcup_{v \in V}\mathcal{V}_v$,
                \`e una biiezione tale per cui $\forall$ $ e = (u, v)$, $dec_E(e) = \mathcal{E}_e \subseteq$ $\{(a, b)$ $\mid$ $a \in \mathcal{V}_u$ $\wedge$
                $b \in \mathcal{V}_v\}$ \`e un insieme di archi rappresentati dal superarco $e$.
        \end{itemize}
    \end{definition}

    Nella definzione, cos\`{i} come d'ora in avanti, si utilizzer\`a la notazione $\mathcal{V}$ e $\mathcal{E}$ per
    indicare, rispettivamente, insiemi di supernodi e superarchi ottenibili attraverso decontrazioni e, quindi, di
    livello inferiore rispetto al grafo decontraito di riferimento.
    Nello specifico, le notazioni $\mathcal{V}_v$ e $\mathcal{E}_v$ saranno utilizzate per indicare, rispettivamente,
    insiemi di nodi e archi di grafi ottenibili attraverso la decontrazione di un certo supernodo $v$.
    Nel contesto di un determinato grafo decontraibile, per portare maggiore distinzione si continuer\`a ad utilizzare
    il termine di nodo ed arco per riferirsi ai supenodi e superarchi di tale livello inferiore. \newline

    Si noti che \`e possibile usare una notazione basata su attributi alternativa a quella delle funzioni per
    descrivere le propriet\`a caratteristiche di nodi ed archi che rendono tale un grafo decontrabile.
    Tale notazione sar\`a utilizzata in seguito per semplificare la descrizione di algoritmi. \newline
    In particolare, si pu\`o definire un grafo decontraibile come un normale grafo diretto sotto forma di coppia
    $(V, E)$ dove:
    \begin{itemize}
        \item $\forall$ $v \in V,$ \`e definito un attributo $v.dec = G_v$ dove $G_v = (\mathcal{V}_v, \mathcal{E}_v)$ \`e un
            grafo decontraibile rappresentato da $v$, quindi tale per cui $dec_V(v) = v.dec$.
        \item $\forall$ $e=(u, v)$  $\in E$, \`e definito una attributo $e.dec = E_e$ dove
            $\mathcal{E}_e \subseteq$ $\{(a, b)$ $\mid$ $a \in \mathcal{V}_u$ $\wedge$ $b \in \mathcal{V}_v\}$ \`e un insieme di archi
            rappresentati da $e$, quindi tale per cui $dec_E(e) = e.dec$.
    \end{itemize}

    Si consideri la natura ricorsiva definizione di grafo decontraibile, per cui se un supernodo $v$ pu\`o essere
    rappresentato da un grafo decontraibile $G_v$, allora i nodi di $G_v$ saranno a loro volta dei supernodi.
    Stessa cosa vale per gli archi, che possono essere rappresentati da insiemi di archi tra i nodi dei grafi.
    La scelta di rendere il grafo decontraibile una struttura ricorsiva, cos\`{i} come la definizione in se stessa,
    \`e utile alle successive definizoni legate ai grafi multi-livello. \newline

    \begin{figure}[h!]
      \centering
      \begin{tikzpicture}
  [mynode/.style={draw, thick, circle, size=0.3mm},
    myarrow/.style={thick, -Triangle},
    ->,shorten >=1pt,auto,node distance=2cm, thick,main node/.style={circle,draw}]

  % Nodes
  \node[main node] (A) {v$_1$};
  \node[main node, blue] (B) [right of=A] {v$_2$};
  \node[main node, blue] (C) [below right of=B] {v$_3$};
  \node[main node] (D) [below left of=C] {v$_4$};

  % Edges
  \path[every node/.style={font=\sffamily\small}];
  \draw[myarrow](A) to node [above] {$e_1$} (B);
  \draw[myarrow, blue] (B) to node [above, name=e2] {$e_2$} (C);
  \draw[myarrow](C) to node [above] {$e_3$} (D);
  \draw[myarrow](B) to node [left] {$e_4$} (D);

  % G_b graph
  \begin{scope}[shift={(6,1)}]
  \draw[blue] (0,0) circle (1.5cm);
  \node[main node] (X) at (-0.5,0.5) {a$_1$};
  \node[main node] (Y) at (1,0) {a$_2$};
  \node[main node] (Z) at (0,-1) {a$_3$};
  \draw[myarrow] (X) -- (Y);
  \draw[myarrow] (Y) -- (Z);
  \draw[myarrow] (Z) -- (X);
  \end{scope}

  % G_c graph
  \begin{scope}[shift={(8,-2.5)}]
  \draw[blue] (0,0) circle (1.5cm);
  \node[main node] (T) at (-0.5,0.5) {a$_4$};
  \node[main node] (U) at (1,0) {a$_5$};
  \node[main node] (V) at (0.25,-1) {a$_6$};
  \node[main node] (W) at (-0.5,-0.5) {a$_7$};
  \draw[myarrow] (U) -- (T);
  \draw[myarrow] (U) -- (W);
  \draw[myarrow] (U) -- (V);
  \end{scope}

  % edges between graphs
  \draw[myarrow, blue] (Z) -- (T);
  \draw[myarrow, blue] (Y) -- (U);

  % Links
  \draw[dashed, line width=1.5pt, red] (B) to[out=65, in=145] node {$dec_{V}(v_2)$} (4.5,1);
  \draw[dashed, line width=1.5pt, red] (e2) to[out=0, in=155] node [below] {$dec_{E}(e_2)$} (6.75,-1);
  \draw[dashed, line width=1.5pt, red] (e2) to[out=0, in=175] (8,-0.7);
  \draw[dashed, line width=1.5pt, red] (C) to[out=300, in=145] node [below] {$dec_{V}(v_3)$} (6.5,-2);
\end{tikzpicture}
      \caption{Un esempio di decontrazione locale di un grafo decontraibile}
      \label{fig:dec-graph-example}
    \end{figure}

    In Figura~\ref{fig:dec-graph-example} \`e mostrato un esempio di grafo decontraibile sulla sinistra, mentre sulla
    destra sono rappresentati i grafi associati ai supernodi $v_2$ e $v_3$, assieme all'insieme di archi associato
    al super-arco $e_2$.
    Si osservi che il grafo $dec_V(v_2)$ composto dai nodi $a_1$, $a_2$ e $a_3$, ad esempio, manca degli archi collegati
    esternamente che definiscono il contesto in cui tale grafo si colloca, e solo uno degli archi incidenti
    in $v_1$ \`e stato espanso.
    Questo fornisce una visione parziale dell'informazione contenuta nel grafo decontraibile a sinistra,
    e per questo si pu\`o dire che il grafo a destra \`e il risultato di un'espansione locale. \newline

    Inoltre, dal momento in cui i grafi decontraibili sono a tutti gli effetti dei grafi, tutte le
    definizioni date sui grafi standard continuano ad essere utilizzate in modo equivalente per i grafi decontraibili.
    Analogamente, un supernodo pu\`o essere considerato come un particolare tipo di nodo, e lo stesso vale per i
    superarchi.
    In particolare, il concetto di isomorfismo pu\`o essere esteso ai grafi decontraibili senza particolari modifiche,
    ignorando l'aspetto \"decontraibile\" di nodi e archi, ovvero le funzioni $dec_V$ e $dec_E$ di entrambi i grafi
    di cui si vuole valutare l'isomofismo.

    % Potrebbe essere altres\`{i} utile indicare un tipo di isomorfismo specifico per i grafi decontraibili, che tenga
    % conto delle funzioni di decontrazione, ovvero che imponga un isomorfismo anche tra i grafi risultanti dalle
    % decontrazioni dei singoli nodi, risultando in una forma di isomorfismo pi\`u forte.

    % \begin{defintion}[Isomorfismo forte di Grafi Decontraibili] \newline
    %    Siano $G = (V, E)$ e $H = (W, F)$ due grafi decontraibili, essi si dicono \textbf{fortemente isomorfi} se
    %     esiste una biiezione $f : V \rightarrow W$ tale per cui
    %     \begin{itemize}
    %        \item $(u, v) \in E$ se e solo se $(f(u), f(v)) \in F$ per ogni $u, v \in V$
    %        \item $dec_V(u)$ \`e fortemente isomorfo a $dec_V(f(u))$ per ogni $u \in V$
    %        \item
    %    \end{itemize}

    %\end{defintion}

    \nlparagraph{Contrazioni di grafi decontraibili}\label{subsec:contrazioni}
    A partire dalla decontrazione, che rappresenta un aspetto intrinseco alla definzione dei grafi decontrabili,
    la relazione di contrazione \`e quella che, intuitivamente, permette di legare grafi decontraibili nel senso
    opposto.
    Se da un lato la decontrazione permette di costruire grafi che rappresentino espansioni locali, con una
    definizone di contrazione si vogliono stabilire le condizioni che permettono di legare interi grafi decontraibili
    ad altri che ne forniscano una loro rappresentazione astratta, affinch\`e la tale rappresentazione
    sia coerente con il concetto di contrazione esistente nella teoria dei grafi.

    \begin{definition}[Contrazione di un Grafo Decontraibile]
        Sia $G = (V, E, dec_V, dec_E)$ un grafo decontraibile, il grafo decontraibile
        $G\mathcal{'} = (\mathfrak{V}, \mathfrak{E}, dec_{\mathfrak{V}}, dec_{\mathfrak{E}})$ \`e una sua
        \textbf{contrazione} se e solamente se:
            \begin{itemize}
                \item l'insieme $\{V_\alpha \mid \alpha \in \mathfrak{V}\}$ \`e una partizione di $V$
                \item l'insieme $(\{E_\alpha \mid \alpha \in \mathfrak{V}\} \setminus \{ \emptyset \}) \cup
                    \{ dec_{\mathfrak{E}}(\epsilon) \mid \epsilon \in \mathfrak{E}\}$ \`e una partizione di $E$.
            \end{itemize}
    \end{definition}

    Nella definzione, cos\`{i} come d'ora in avanti, si utilizzer\`a la notazione $\mathfrak{V}$ e $\mathfrak{E}$ per
    indicare, rispettivamente, gli insiemi di supernodi e superarchi di un grafo contratto, e quindi di livello
    superiore, rispetto al grafo di riferimento. \newline

    Nella figura~\ref{fig:contraction-example}, sulla sinistra \`e mostrato un esempio di grafo decontraibile
    contrazione del grafo decontraibile a destra, dove sono annotati per ogni supernodo e superarco
    i nodi e gli archi ottenibili dalle loro decontrazioni. \newline

    \begin{figure}
        \centering
        \include{TikZPictures/contraction_example}
        \caption{Esempio di contrazione di un grafo decontraibile}
        \label{fig:contraction-example}
    \end{figure}

    Dalla definizione si evince che le seguenti sono condizioni necessarie affinch\`e un grafo decontraibile $G'$ possa
    essere una contrazione di un grafo decontraibile $G$:
    \begin{enumerate}[(i)]
        \item I nodi di $G'$ devono avere tutti un grafo non vuoto come decontrazione, ovvero $V_\alpha \neq \emptyset$
        per ogni $\alpha \in \mathfrak{V}$.
        Infatti se fosse che $V_\alpha = \emptyset$ per qualche $\alpha \in \mathfrak{V}$, allora l'insieme
        $\{V_\alpha \mid \alpha \in \mathfrak{V}\}$ non costituirebbe una partizione di $V$, in quanto, per definzione,
        l'insieme vuoto non pu\`o essere incluso in una partizione.
        \item Gli insiemi di nodi delle decontrazioni dei supernodi in $G'$ devono essere a due a due disgiunti, ovvero
        non possono esistere nodi a cui corrispondono contemporaneamente due supernodi distinti, in quanto anche in
        questo caso l'insieme $\{V_\alpha \mid \alpha \in \mathfrak{V}\}$ non costituirebbe una partizione di $V$.
        \item I superarchi di $G'$ devono avere tutti una decontrazione non vuota.
        Infatti, se fosse che $dec_{\mathfrak{E}}(\epsilon) = \emptyset$ per qualche $\epsilon \in \mathfrak{E}$, allora
        l' insieme $(\{E_\alpha \mid \alpha \in \mathfrak{V}\} \setminus \{ \emptyset \}) \cup
        \{ dec_{\mathfrak{E}}(\epsilon) \mid \epsilon \in \mathfrak{E}\}$ non costituirebbe una partizione di $E$,
        analogamente a quanto detto nel primo punto. \newline
        Si noti che la condizione di disgiunzione tra gli insiemi di archi delle decontrazioni dei superarchi \`e
        automaticamente soddisfatta dalla definzione di grafo decontraibile e delle decontrazioni dei suoi archi.
    \end{enumerate}

    E' rilevante notare che, data una contrazione $G'$ del grafo decontraibile $G$, essa contiene tutte le informazioni
    necessarie a calcolare la struttura di archi e nodi di $G$.
    Infatti, sia $G' = (\mathfrak{V}, \mathfrak{E})$ una contrazione di $G$, allora dalla definizione si ha:

    \begin{equation*}
        G = (\bigcup_{\alpha \in \mathfrak{V}} V_\alpha ,
            (\bigcup_{\alpha \in \mathfrak{V}} E_\alpha \cup \bigcup_{\epsilon \in \mathfrak{E}}{dec_{\mathfrak{E}}(\epsilon)}),
            dev_V, dec_E) %TODO: \bigsqcup{\alpha \in \mathfrak{V}} dec_{V_{\alpha}}, (\bigsqcup{\alpha \in \mathfrak{V}} dec_{E_{\alpha}} \cup ???)
    \end{equation*}

    dove $dec_V$ è ottenuto dall'unione delle funzioni $dec_{\mathfrak{V}_v}$ per ciascun $v \in \mathfrak{V}$,
    e $dec_E$ è ottenuto dall'unione delle funzioni $dec_{\mathfrak{E}_e}$ per ciascun $e \in \mathfrak{E}$ combinato con
    tutte le mappature tra superarchi in $\bigcup{\epsilon \in \mathfrak{E}} dec_{\mathfrak{E}}(\epsilon)$ e le loro
    decontrazioni.
    Definiamo, quindi, l'operatore unario $.^D : \mathcal{G}_D \rightarrow \mathcal{G}_D$ come l'operatore di
    \textbf{decontrazione completa} che, dato un grafo decontraibile $G = (V, E)$ restituisce il grafo decontrabile
    $G^D$ ottenuto dalla decontrazione di tutti i supernodi e superarchi del grafo in input.

    \begin{equation*}
        G^D = (\bigcup_{v \in V} \mathcal{V}_v , (\bigcup_{v \in V} \mathcal{E}_v \cup \bigcup_{e \in E} dec_E(e)),
        dec_{\mathcal{V}}, dec_{\mathcal{E}})
    \end{equation*}

    Una contrazione di un grafo decontraibile $G$, pu\`o quindi essere alternativamente definta come un suo grafo
    quoziente decontrabile $G'$ la cui decontrazione completa $(G')^D$ \`e proprio $G$.
    \newline

    Si noti che, in generale, un grafo decontraibile $G$ pu\`o non essere una contrazione di $G^D$.
    Questo pu\`o verificarsi unicamente quando $G$ non soddisfa tutte le condizioni (i), (ii) e (iii) presentate in
    precendenza. \newline
    Per rendere chiaro questo aspetto, in figura~\ref{fig:non-contraction-example} \`e proposta una variazione
    dell'esempio precedente, in cui il grafo decontraibile a sinistra, ottenuto come decontrazione completa del grafo
    destra, non \`e una sua contrazione, in quanto il super-nodo $a_3$ appartiene contemporaneamente a $V_a$ e $V_b$,
    e il super-arco $e_3$ viene decontratto in un insieme vuoto di archi, violando le condizioni (ii) e (iii).

    \begin{figure}
        \centering
        \begin{tikzpicture} [mynode/.style={draw, thick, circle, minimum size=0.3cm},
    ->,>={stealth},shorten >=1pt,auto,node distance=2cm,
    thick,main node/.style={circle,draw}]

    % Nodes
    \node[main node,
        label={[align=center, blue, font=\tiny]above:{$V_{v_1} = \{a_1, a_2, a_3\}$}\\{$E_a = \emptyset$}}]
    (A) {v$_1$};
    \node[main node,
        label={[align=center, blue, font=\tiny]above:{$V_{v_2} = \{a_3, a_4, a_5\}$}\\{$E_{v_2} = \{(a_3, a_4),$}\\{$(a_4, a_5), (a_5, a_3)\}$}}]
    (B) [right of=A, node distance=2.5cm] {v$_2$};
    \node[main node,
        label={[align=left, blue, font=\tiny]right:{$V_{v_3} = \{a_6, a_7, a_8, a_9\}$}\\{$E_{v_3} = \{(a_7, a_6),(a_7, a_9),$}\\{$(a_7, a_8)\}$}}]
    (C) [below right of=B] {v$_3$};
    \node[main node,
        label={[align=center, blue, font=\tiny]left:{$V_{v_4} = \{a_{10}\}$}\\{$E_{v_4} = \emptyset$}}]
    (D) [below left of=C] {v$_4$};

    % Edges
    \path[every node/.style={font=\sffamily\small}];
    \draw[myarrow](A) to node [above,
    label={[align=center, blue, font=\tiny]below:{$E_{e_1} = $}\\{$\{(a_2, a_5)\}$}}] {$e_1$} (B);
    \draw[myarrow](B) to node [above,
    label={[align=left, blue, font=\tiny]right:{$E_{e_2} = \{(a_5,a_6), (a_4, a_7)\}$}}] {$e_2$} (C);
    \draw[myarrow](C) to node [below,
    label={[align=left, blue, font=\tiny]right:{$E_{e_3} = \emptyset$}}] {$e_3$} (D);
    \draw[myarrow](B) to node [right, label={[align=right, blue, font=\tiny]left:{$E_{e_4} = \{(a_5,a_{10})\}$}}] {$e_4$} (D);

    % G_b graph
    \begin{scope}[shift={(8.5,1)}]
    \draw[blue] (0,0) circle (1.5cm);
    \node[above, blue] at (0,1.5) {$v_2$};
    \node[main node] (X) at (-1,0) {$a_3$};
    \node[main node] (Y) at (1,0) {$a_4$};
    \node[main node] (Z) at (0,-1) {$a_5$};
    \draw[myarrow] (X) -- (Y);
    \draw[myarrow] (Y) -- (Z);
    \draw[myarrow] (Z) -- (X);
    \end{scope}

    % G_c graph
    \begin{scope}[shift={(10.5,-1.75)}]
    \draw[blue] (0,0) circle (1.5cm);
    \node[right, blue] at (1.5,0) {$v_3$};
    \node[main node] (T) at (-0.5,0.5) {$a_6$};
    \node[main node] (U) at (1,0) {$a_7$};
    \node[main node] (V) at (0.25,-1) {$a_8$};
    \node[main node] (W) at (-0.5,-0.5) {$a_9$};
    \draw[myarrow] (U) -- (T);
    \draw[myarrow] (U) -- (W);
    \draw[myarrow] (U) -- (V);
    \end{scope}

    % G_a graph
    \begin{scope}[shift={(6.7, 1)}]
    \draw[blue] (0,0) circle (1.2cm);
    \node[above, blue] at (0,1.2) {$v_1$};
    \node[main node] (N) at (-0.4,0.5) {$a_1$};
    \node[main node] (M) at (-0.4,-0.5) {$a_2$};
    \end{scope}

    % G_d graph
    \begin{scope}[shift={(7.5,-2.5)}]
    \draw[blue] (0,0) circle (0.75cm);
    \node[left, blue] at (-0.75,0) {$v_4$};
    \node[main node] (K) at (0,0) {$a_{10}$};
    \end{scope}

    % edges between graphs
    \draw[myarrow, blue] (M) -- (Z);
    \draw[myarrow, blue] (Z) -- (K);
    \draw[myarrow, blue] (Z) -- (T);
    \draw[myarrow, blue] (Y) -- (U);
\end{tikzpicture}
        \caption{Esempio di grafo decontraibile che non \`e contrazione della sua decontrazione completa}
        \label{fig:non-contraction-example}
    \end{figure}

    Sarebbe, quindi, scorretto dire che un grafo $G'$ \`e una contrazione di $G$ se $(G')^D = G$.
    In particolare, considerato quanto gi\`a detto, si pu\`o facilmente dimostrare la seguente proposizione.

    \begin{proposition}
        Siano $G = (V, E)$ e $G' = (\mathfrak{V}, \mathfrak{E})$ due grafi decontraibili. Allora:
        \begin{equation*}
            \left\{
            \begin{aligned}
                &V_\alpha \neq \emptyset  &&\forall \alpha \in \mathfrak{V} \\
                &V_{\alpha} \cap V_{\beta} = \emptyset &&\forall \alpha, \beta \in \mathfrak{V}, \, \alpha \neq \beta \\
                &E_{\epsilon} \neq \emptyset  &&\forall \epsilon \in \mathfrak{E}
            \end{aligned}
            \right\}
            \land G = (G')^D \quad \Longleftrightarrow \quad G' \text{ è una contrazione di } G
        \end{equation*}
    \end{proposition}

    In aggiunta alla precendente proposizione, grazie alla definizione di contrazione tra grafi decontraibili,
    si propone un'altra proposizione che descrive come il concetto di grafo indotto pu\`o essere usato per descrivere
    le decontrazioni.

    \begin{proposition}
    Sia $G=(V, E)$ un grafo decontraibile e sia $G' = (\mathfrak{V}, \mathfrak{E})$ una sua contrazione,
    sia $\alpha$ un super-nodo appartenente a $\mathfrak{V}$.
    Allora $dec_{\mathfrak{V}}(\alpha) = (V_\alpha, E_\alpha)$ \`e il sottografo di $G$ indotto da $V_\alpha$.
    \begin{equation*}
        dec_{\mathfrak{V}}(\alpha) = G[V_\alpha]
    \end{equation*}
    \end{proposition}

    \paragraph{Dimostrazione}
    Sia $H = (W, F)$ il sottografo di $G = (V, E)$ indotto da $V_\alpha$ con
    $\alpha \in \mathfrak{V}$, per definizione di grafo indotto, $H$ deve essere definito sull'insieme di nodi
    $V_\alpha$, ovvero deve essere $W = V_\alpha$.
    Si vuole ora dimostrare che $F = E_\alpha$. \newline

    L'inclusione $F \subseteq E_\alpha$ pu\`o essere dimostrata notando che per definizione $H$,
    che \`e un grafo indotto da $V_\alpha$, si ha:
    \begin{equation*}
    (x, y) \in F \implies (x, y) \in E \quad \text{con} \quad x, y \in V_{\alpha}
    \end{equation*}
    Essendo che $\{ E_\beta \mid \beta \in \mathfrak{V}\}
    \cup \{ dec_{\mathfrak{E}}(\epsilon) \mid \epsilon \in \mathfrak{E}\}$ \`e un ricoprimento di $E$, si nota che
    l'unico insieme del ricoprimento che pu\`o contenere archi definiti in $V_\alpha \times V_\alpha$
    \`e proprio $E_\alpha$. Si conclude allora $(x, y) \in F \implies (x, y) \in E \implies (x, y) \in E_\alpha $.
    \newline

    L'inclusione $E_\alpha \subseteq F$ pu\`o essere dimostrata notando che per la propriet\`a delle contrazioni,
    per cui $\{ E_\beta \mid \beta \in \mathfrak{V}\} \cup
    \{dec_{\mathfrak{E}}(\epsilon) \mid \epsilon \in \mathfrak{E}\} $ \`e una copertura di $E$, si ha:
    \begin{equation*}
        (u, v) \in E_\alpha \implies (u, v) \in E
    \end{equation*}
    Essendo che  $(u, v) \in E_\alpha \implies (u, v) \in V_\alpha \times V_\alpha$ per definzione di $E_\alpha$, si
    deve avere $u, v \in V_\alpha$. Segue quindi $(u, v) \in E_\alpha \land u,v \in V_\alpha \implies (u, v) \in F$.

    \newpage

    \subsection{Grafo multi-livello}\label{subsec:grafo-multi-livello}

A partire dalla definizione di grafo decontraibile e di contrazione, pu\`o essere definita una struttura gerarchica
    a pi\`u livelli di grafi decontraibili che siano l'uno la contrazione dell'altro, dove i grafi ai livelli inferiori
    o loro sottografi possano essere ottenuti attraverso, rispettivamente, decontrazioni complete o espansioni locali
    dei livelli superiori.

    \nlparagraph{Funzioni}

    \begin{definition} [Funzione di contrazione]
    Una \textbf{funzione di contrazione} $f_C : \mathcal{G}_D \rightarrow \mathcal{G}_D$ \`e una funzione che dato un
    grafo decontraibile $G$ essa produce un nuovo grafo decontraibile $f_C(G) = G'$ che sia una contrazione di $G$.
    \end{definition}

    Una funzione di contrazione rappresenta quindi un particolare schema di contrazione dove dominio e codominio sono
    coincidenti e sono rappresentati dall'insieme dei grafi decontraibili $\mathcal{G}_D$. Essa produce contrazioni
    dei grafi decontraibili in input secondo determinate logiche definite dalla funzione stessa.
    Nel corso di questo capitolo, i termini \textit{funzione di contrazione} e \textit{schema di contrazione}
    saranno utilizzati in modo intercambiabile. \newline
    \'E importante notare il fatto che la funzione sia chiusa rispetto all'insieme dei grafi decontraibili: questo vuol
    dire che \`e possibile comporre pi\`u funzioni di contrazione in sequenza a partire da un dato grafo decontraibile.
    \newline

    \begin{definition} [Funzione di trasformazione naturale]
    Definiamo \textbf{funzione di trasformazione naturale} $\eta$ una funzione che dato un grafo standard
        $H=(W,F)$, produce il corrispondente grafo decontraibile $G = (V, E, dec_V, dec_E)$ con le seguenti propriet\`a:
        \begin{itemize}
            \item $dec_V(v) = (\emptyset, \emptyset)$ \quad $\forall v\in V$
            \item $dec_E(e) = \emptyset$ \quad $\forall e\in E$
            \item $H$ e $G$ sono isomorfi
        \end{itemize}
    \end{definition}

    La funzione trasformazione naturale \`e quindi la funzione che permette di trasformare un dato grafo standard in un
    grafo decontraibile isomorfo per cui le due funzioni di decontrazione dei nodi e degli archi siano definite,
    seppur producano rispettivamente un grafo e un insieme di archi vuoto. \newline
    Si pu\`o osservare che queste propriet\`a garantiscono che il grafo decontraibile ottenuto non possa essere
    contrazione di alcun altro grafo decontraibile. \newline

    Il nome di funzione di trasformazione naturale deriva dalla teoria delle categorie, dove una trasformazione
    naturale \`e una mappa tra due funtori che preserva la struttura tra i due oggetti e la coerenza delle operazioni
    che possono essere applicate su di essi. Sebbene in matematica i grafi non siano considerati essi stessi funtori,
    nella programmazione funzionale \`e possibile che i grafi vengano interpretati come tali. \newline

    \nlparagraph{Definizione dei Grafi Multi-Livello}\label{subsec:definzione-grafi_multilivello}

    Il concetto di grafo multi-livello \`e di seguito definito attraverso una decrizione bottom-up di tale struttura,
    indicandone il grafo inziale e gli schemi di contrazione, che descrivono il modo in cui dei sottografi di un
    determinato livello sono collassati in singoli supernodi del livello superiore, formando gerarchie di grafi
    decontraibili.

    \begin{definition}[Grafo multi-livello]
    Un \textbf{grafo multi-livello} $M$ \`e una coppia $(G, \Gamma)$ dove:
        \begin{itemize}
            \item $G = (V, E)$ \`e un grafo
            \item $\Gamma$ \`e una sequenza $\langle f_{C_1}, f_{C_2}, .., f_{C_k} \rangle$ di funzioni di contrazione
        \end{itemize}
    \end{definition}

    Utilizzando la notazione $G_i$ per indicare il grafo decontraibile collocato al livello $i$-esimo della gerarchia,
    si pu\`o consierare il grafo multi-livello $M$ come una sequenza di grafi decontraibili
    $\langle G_0, G_1, .., G_k \rangle$, dove il grafo $G_0 = \eta(G)$ pu\`o essere ottenuto dalla funzione di
    trasformazione naturale $\eta$ applicata al grafo standard $G$.
    Per questo, le funzioni di contrazione di un grafo multi-livello dovono essere tali che
    $f_{C_i}(G_{i-1}) = G_i$ per ogni $i \in \{1, .., k\}$. \newline

    Dato un grafo multi-livello $M = (G,\Gamma)$ con $\Gamma = \langle f_{C_1}, f_{C_2}, .., f_{C_k} \rangle$,
    la funzione che calcola il suo grafo decontraibile al livello k-esimo $G_k$ pu\`o quindi essere descritta
    attraverso la seguente definizione ricorsiva: \newline
    \begin{equation*}
        con(M, k) =
        \left\{
        \begin{aligned}
            &f_{C_k}(con(M, k-1)) && \text{se } k > 0\\
            &\eta(G)  && \text{se } k = 0
        \end{aligned}
        \right\}
        = G_k
    \end{equation*} \newline

    Si pu\`o notare che la funzione da applicare a $G$ per ottenere $G_{k}$ dovr\`a essere la composizione ordinata
    delle funzioni di contrazione in $\Gamma$ fino al livello $k$ abbinata alla funzione di contrazione $\eta$ per
    ottenere $G_0$, ovvero:
    \begin{equation*}
        G_k = (f_{C_k} \circ f_{C_{k-1}} \circ \ldots \circ f_{C_1} \circ \eta)(G)
    \end{equation*}

    Sia $\mathcal{M}$ l'insieme dei grafi decontraibili, definiamo, inoltre, la funzione altezza $h$, sia
    per grafi multi-livello che per i suoi grafi decontraibili, nel modo seguente:

    \begin{itemize}
        \item $h : \mathcal{M} \rightarrow \mathbb{N}$, tale che $h(M) = k$, con $M = (G, \Gamma)$ e $k$ il numero di
        funzioni di contrazione in $\Gamma$. Ad esempio $h(M) = 0$ se $M = (G, \langle \rangle)$.
        \item $h : \mathcal{G}_D \rightarrow \mathbb{N}$, tale che $h(G_i) = i$, con $i$ il numero di contrazioni
        necessarie per ottenere $G_i$ a partire da $\eta(G)$. Ad esempio $h(\eta(G)) = 0$.
    \end{itemize}

    In figura~\ref{fig:multi-level-graph-example} \`e mostrato un esempio di grafo multi-livello di altezza 2,
    rappresentato mediante una sequenza di grafi decontraibili $G_0, G_1,$ e $G_2$. I grafi dei livelli superiori
    sono ottenuti rispettivamente attraverso funzioni di contrazione per cricche (contrazione da $G_0$ a $G_1$) e
    per componenti connesse (contrazione da $G_1$ a $G_2$).

    \begin{figure}
        \begin{tikzpicture}[x={(1cm,0cm)},y={(0cm,1cm)},z={(0.410cm,0.300cm)}]
    \node[canvas is zy plane at x=0,draw,fill=white] at (0,0) {
    \includegraphics[scale=0.315]{Immagini/graph0.png}
    };

    \node[canvas is zy plane at x=5,draw,fill=white] at (0,0) {
        \includegraphics[scale=0.315]{Immagini/graph1.png}
    };

    \node[canvas is zy plane at x=10,draw,fill=white] at (0,0) {
        \includegraphics[scale=0.315]{Immagini/graph2.png}
    };
\end{tikzpicture}
        \caption{Esempio di grafo multilivello}
        \label{fig:multi-level-graph-example}
    \end{figure}

    \newpage

    \nlparagraph{Algoritmo di trasformazione naturale}\label{subsec:algoritmo-di-trasformazione-naturale}

    La generica procedura algoritmica per realizzare la trasformazione naturale di un grafo standard $H = (W, F)$
    preso in input in un grafo decontraibile $G = (V, E)$, può essere definita considerando la creazione di supernodi
    e superarchi, costruendo implicitamente la funzione biiettiva $f_V: W \rightarrow V$ che realizza l'isomorfismo tra i nodi
    di $H$ e i supernodi di $G$.

    \begin{algorithm}[H]
    \begin{algorithmic}[1]
        \caption{NATURAL-TRANSFORMATION($H$)}\label{alg:cap1}
        \State $V$=$\emptyset$
        \For{ each $x \in W$}
            \State let $v$ be a super-node such that $f_V(x)=v$
            \State $v.dec=(\emptyset, \emptyset)$
            \State $V = V \cup \{v\}$
        \EndFor
        \State $E$=$\emptyset$
        \For{ each $g=(x,y) \in F$}
            \State let $e$ be a super-edge such that $e=(f_V(x), f_V(y))$
            \State $e.dec=\emptyset$
            \State $E=E\cup \{e\}$
        \EndFor
        \State \textbf{return} $(V, E)$
    \end{algorithmic}
\end{algorithm}

    Tramite l'ausilio di strutture dati con tempi di ricerca costanti, come insiemi di hash, la complessit\`a delle
    operazioni all'interno dei due circli \`e $O(1)$,
    e, di conseguenza, la complessit\`a dell'algoritmo \`e dettata dal numero di nodi e archi del grafo in input $H$,
    ovvero $\Theta(|W| + |F|)$.