\section{Procedure di Contrazione}\label{cap:procedure-contrazione}

In questa sezione si entrerà nel merito delle dinamiche interne alla costruzione della struttura di un grafo
multi-livello e alle sue funzioni di contrazione.
Sebbene nella Sezione~\ref{sec:algoritmi-di-enumerazione} siano già stati illustrati gli
algoritmi specifici per il riconoscimento e l'elencazione di pattern strutturali all'interno di un grafo,
nulla è stato ancora detto di come queste procedure di enumerazione debbano essere sfruttate nel contesto più
complesso del grafo multi-livello.
In particolare, si vuole dare una visione più chiara di come sia algoritmicamente
possibile realizzare una rete di collegamenti tra supernodi su più livelli attenendosi alle definizioni
fornite nella Sezione~\ref{sec:definizioni-e-proprieta-di-base} di questo capitolo, affinché si possa ottenere un tale
risultato a partire da una qualsiasi sequenza di insiemi di nodi generati dalle procedure di enumerazione,
evidenziandone complessità, problematiche e possibili soluzioni.

\subsection{Riduzione disgiunta}

Si vuole ora considerare il generico problema per cui, dato un grafo decontraibile $G = (V, E)$ e un insieme di
sottoinsiemi di nodi $Q \subseteq \mathcal{P}(V)$ che costituisce una copertura di $V$, si vuole costruire un grafo
decontraibile $G'$ che rappresenti l'informazione fornita dal raggruppamento dei nodi descritto in $Q$.
Aspetto di fondamentale importanza è che l'insieme $Q$ non costituisce necessariamente una partizione.
$Q$ potrebbe quindi essere calcolato a partire da un grafo $G$ attraverso un algoritmo di enumerazione, in modo tale
che i suoi elementi siano insiemi di nodi appartenenti ad un certo pattern strutturale.
Tuttavia, come si può immaginare, tali insiemi non sono necessariamente disgiunti: basti pensare alle cricche e ai
circuiti semplici, che possono avere nodi in intersezioni di più insiemi.
Le componenti fortemente connesse invece, essendo costruite a partire da una relazione di equivalenza, sono un
valido esempio di pattern strutturale che costituisce sempre una partizione di $V$.

D'ora in avanti, ci riferiremo agli elementi di una tale copertura di nodi $Q$ (non necessariamente disgiunta)
come \textbf{insiemi componente}. \newline

Per mantenere le proprietà delle contrazioni su grafi è però necessario che ad ogni nodo corrisponda ad uno ed un solo
supernodo.
La presenza dello stesso nodo in più supernodi contratti non sarebbe coerente con la teoria dei grafi
già affermata, oltre che portare ad un risultato che rischia di avere limitate applicabilità.
Si vuole quindi definire una procedura per rappresentare l'informazione fornita da $Q$ in un nuovo insieme che sia una
partizione di $V$ rappresentativa di $Q$ e, a partire da questa, costruire un grafo decontraibile $G'$.
Abbinare questa definizione generica, valida nella teoria degli insiemi, ad uno specifico algoritmo per calcolare
$Q$ a partire dalla struttura di un grafo, è il primo passo che permetterà di definire particolari funzioni di
contrazione.

\nlparagraph{Definzione di riduzione disgiunta}

\begin{definition}[Riduzione disgiunta]
Sia $V$ un insieme di elementi. Sia $Q \subseteq \mathcal{P}(V)$ una copertura di $V$.
Definiamo la \textbf{riduzione disgiunta} di $Q$, e la indichiamo con $\mathcal{D}(Q)$,
l'insieme di insiemi di elementi in $V$ tale per cui:
\begin{enumerate}[(i)]
    \item $\forall A \in \mathcal{D}(Q), \quad u, v \in A \Leftrightarrow \{C \in Q \mid u \in C\} = \{C \in Q \mid v \in C\}$
    \item $\emptyset \notin \mathcal{D}(Q)$
\end{enumerate}
\end{definition}

Una riduzione disgiunta di una copertura $Q$ è quindi un insieme di insiemi di nodi in $V$ tale per cui due
nodi $u$ e $v$ appartengono allo stesso insieme in $\mathcal{D}(Q)$ se e solo se sono inclusi nella stessa combinazione
di insiemi in $Q$.

\begin{figure}[!h] \centering
\resizebox{!}{5.3cm}{
    \begin{tikzpicture}[state/.style ={circle, draw, color=black , fill=blue, text=white, inner sep=0cm,
    minimum size=0.1cm}]
        \node[state] (1) [label={[left,text=blue]:$v_1$}] at (-0.3, 0) {};
        \node[state] (2) [label={[left,text=blue]:$v_2$}] at (1, 0.5) {};
        \node[state] (3) [label={[left,text=blue]:$v_3$}] at (0, -0.6) {};
        \node[state] (4) [label={[right,text=blue]:$v_4$}] at (-0.4, 0.8) {};
        \node[state] (5) [label={[right,text=blue]:$v_5$}] at (0.8, 1.7) {};
        \node[state] (6) [label={[left,text=blue]:$v_6$}] at (1.9, 2.3) {};
        \node[state] (7) [label={[left,text=blue]:$v_7$}] at (4, 0) {};
        \node[state] (8) [label={[left,text=blue]:$v_8$}] at (3.2, -0.8) {};
        \node[state] (9) [label={[right,text=blue]:$v_9$}] at (3.6, -0.8) {};
        \node[state] (10) [label={[right,text=blue]:$v_{10}$}] at (2.5, 0.5) {};
        \node[state] (11) [label={[below,text=blue]:$v_{11}$}] at (4, 1.5) {};
        \node[state] (12) [label={[above,text=blue]:$v_{12}$}] at (4.5, 1.2) {};

        \node[fit={(1)(2)(3)(4)}, draw, circle, label=below:$C_1$]{};
        \node[fit={(5)(6)}, draw, circle, label=left:$C_5$]{};
        \node[fit={(7)(10)},draw, ellipse, label=right:$C_4$, minimum width = 2.7cm]{};
        \node[fit={(2)(10)},draw, ellipse, label=below:$C_3$, minimum width = 2.7cm, minimum height = 1cm]{};
        \node[fit={(7)(8)(9)(10)(11)(12)}, draw, circle, label=above:$C_2$]{};
        \node[fit={(11)(12)}, draw, circle, label=left:$C_6$]{};
        \node (Q) at (2.3, -2) {$Q$};
    \end{tikzpicture}
    \hspace{0.5cm}
    \begin{tikzpicture}[state/.style ={circle, draw, color=black , fill=blue, text=white, inner sep=0cm,
    minimum size=0.1cm}]
        \node[state] (1) [label={[left,text=blue]:$v_1$}] at (-0.3, 0) {};
        \node[state] (2) [label={[left,text=blue]:$v_2$}] at (1, 0.5) {};
        \node[state] (3) [label={[left,text=blue]:$v_3$}] at (0, -0.6) {};
        \node[state] (4) [label={[right,text=blue]:$v_4$}] at (-0.4, 0.8) {};
        \node[state] (5) [label={[right,text=blue]:$v_5$}] at (0.8, 1.7) {};
        \node[state] (6) [label={[left,text=blue]:$v_6$}] at (1.9, 2.3) {};
        \node[state] (7) [label={[left,text=blue]:$v_7$}] at (4, 0) {};
        \node[state] (8) [label={[left,text=blue]:$v_8$}] at (3.2, -0.8) {};
        \node[state] (9) [label={[right,text=blue]:$v_9$}] at (3.6, -0.8) {};
        \node[state] (10) [label={[right,text=blue]:$v_{10}$}] at (2.5, 0.5) {};
        \node[state] (11) [label={[below,text=blue]:$v_{11}$}] at (4, 1.5) {};
        \node[state] (12) [label={[above,text=blue]:$v_{12}$}] at (4.5, 1.2) {};

        \node[fit={(1)(3)(4)}, draw, ellipse]{};
        \node[fit={(5)(6)}, draw, circle]{};
        \node[fit={(2)}, draw, circle]{};
        \node[fit={(10)}, draw, circle]{};
        \node[fit={(7)}, draw, circle]{};
        \node[fit={(8)(9)}, draw, ellipse, minimum width = 2cm]{};
        \node[fit={(11)(12)}, draw, circle]{};
        \node (DQ) at (2.2, -2) {$D(Q)$};
    \end{tikzpicture}}
\caption{Esempio di riduzione disgiunta}
\label{fig:disjoint_reduction_example}
\end{figure}

L'esempio in Figura~\ref{fig:disjoint_reduction_example} fornisce una rappresentazione di una copertura $Q$ di
elementi (a sinistra) accompagnata dalla sua riduzione disgiunta $D(Q)$ (a destra). \newline

Il nome di questa definizione deriva dal fatto che:
\begin{itemize}
    \item il risultato di questo operatore su una copertura $Q$, da solo, non contiene le informazioni sufficienti a
        descrivere la copertura che lo ha originato, che potrebbero essere molteplici (ma non infinite, se si ha a che
        fare con insiemi finiti). %TODO: nota: potrebbe essere una classe di equivalenza tra coperture!!
        Da qui il termine ``riduzione''.
    \item il risultato di questo operatore su una copertura $Q$ di $V$ costituisce una partizione di $V$, come
        dimostrato in seguito, e fornisce un criterio per una suddivisione degli elementi in $V$ in insiemi disgiunti
        tra loro.
        Da qui il termine ``disgiunta''.
\end{itemize}

\newpage

\begin{proposition}
Sia $V$ un insieme di elementi. Sia $Q \subseteq \mathcal{P}(V)$ una copertura di $V$.
La riduzione disgiunta di $Q$ costituisce una partizione di $V$.
\end{proposition}

\paragraph{Dimostrazione}
Si definisce la relazione $R_Q$ sull'insieme $V$ come
\begin{equation*}
    uR_{Q}v \Leftrightarrow \{C \in Q \mid u \in C\} = \{C \in Q \mid v \in C\}
\end{equation*}

Essa \`e una relazione di equivalenza, in quanto rispetta le propriet\`a di riflessivit\`a, simmetria e transitivit\`a,
che sono date dalle rispettive propriet\`a dell'uguaglianza.
Segue che l'insieme quoziente $V/R_{Q}$ deve rappresentare una partizione degli elementi in $V$.

Per il punto (i) della definizione di $\mathcal{D}(Q)$, i suoi insiemi non vuoti rappresentano le classi di equivalenza
di $R_Q$.
Inoltre per il punto (ii), l'insieme vuoto non pu\`o appartenere a $D(Q)$.
Si conclude allora $D(Q) = V/R_{Q}$, e pertanto $D(Q)$ \`e una partizione di $V$.

\nlparagraph{Definizione tramite ipergrafi}

Un \textit{ipergrafo} è una struttura simile ad un grafo non orientato dove ogni arco, detto \textit{iperarco},
anziché collegare una semplice coppia di nodi può connettere un arbitrario sottoinsieme di vertici.
Più formalmente, un ipergrafo $H$ è una coppia $(X, E)$ dove $X$ è un insieme di elementi chiamati nodi ed $E$ è
un insieme di sottoinsiemi di $X$ chiamati iperarchi.
Si consideri ora un ipergrafo non orientato $H = (X, E)$.
Notando che un iperarco non orientato $e \in E$ è a tutti gli effetti un insieme di nodi $v \subseteq X$,
si pu\`o estendere il concetto di rappresentazione disgiunta anche nel dominio degli ipergrafi.

\begin{definition}[Riduzione disgiunta di un ipergrafo]
    Dato un ipergrafo non orientato $H = (X, E)$ tale per cui $E$ \`e un ricoprimento di $X$,
    definiamo \textbf{riduzione disgiunta} di $H$ un nuovo ipergrafo non orientato $J = (W, F)$
    tale per cui $W = V$ e $F = \mathcal{D}(E)$.
\end{definition}

Si pu\`o notare, quindi, che la funzione di riduzione disgiunta pu\`o essere considerata come una
endofunzione idempotente sull'insieme degli ipergrafi non orientati.
\newpage

\nlparagraph{Contrazione costruita da una partizione}

Motivo per cui la riduzione disgiunta di una copertura è certamente utile nel momento in cui si voglia
realizzare una funzione di contrazione, è che essa permette di ottenere una partizione di nodi.
La seguente ulteriore definizione permette di chiarire il passaggio che porta alla produzione in output di un grafo
decontraibile.

\begin{definition}[Contrazione costruita da una partizione]
Sia $G = (V, E, dec_V, dec_E)$ un grafo decontraibile, sia $P$ una partizione di $V$. \\
Si definisce \textbf{contrazione di G costruita sulla partizione P} il grafo decontraibile
$G' = (\mathfrak{V}, \mathfrak{E}, dec_{\mathfrak{V}}, dec_{\mathfrak{E}})$ contrazione di $G$ tale per cui
    \begin{equation*}
        \{V_\alpha \mid \alpha \in \mathfrak{V}\} = P
    \end{equation*}
\end{definition}

Dato un grafo decontraibile $G$ e una sua partizione dei nodi, quindi, è sempre possibile calcolare la sua
contrazione $G'$ contraendo nodi di $G$ in supernodi di $G'$ secondo gli insiemi descritti dalla partizione.
Si noti, infatti, che esiste una biiezione tra l'insieme delle possibili contrazioni di $G$ e l'insieme delle
partizioni di $V$ e che, pertanto, data una partizione di $V$ esiste una ed una sola contrazione di $G$ costruita
su di essa.

\nlparagraph{Algoritmo per la contrazione costruita da una riduzione disgiunta}\label{sec:make_decontractible_graph}

Entrambi i concetti esposti nel paragrafo precedente, ovvero la riduzione disgiunta di una copertura e la
contrazione costruita da una partizione, sono considerabili come elementi sufficientemente generici
per la definizione di una qualsiasi procedura di contrazione.
In particolare, il passaggio da una copertura
di nodi ad un grafo decontraibile può essere considerato come l'operazione successiva all'enumerazione degli
insiemi di nodi che costituiscono la copertura stessa.

Risulta quindi utile definire un algoritmo che a partire da un grafo decontraibile $G$ e una sua copertura $Q$,
fornisca in output la contrazione di $G$ costruita sulla riduzione disgiunta di $Q$. \newline

Una delle possibili rappresentazioni della copertura $Q$, da fornire in input a tale algoritmo, è quella un
\textit{dizionario} (talvolta anche detto ``mappa'' o ``tabella''), una struttura dati astratta che rappresenta una
collezione di coppie di elementi chiave-valore e le cui operazioni fondamentali consistono nell'inserimento e
cancellazione di coppie e ricerca di un valore mediante la sua chiave. \`E importante che in un dizionario
ad ogni chiave corrisponda al più un solo valore.
Negli pseudocodici, dato un dizionario $T$, l'operazione di ricerca del valore associato ad una chiave $k$ verrà
indicata con $T[k]$, mentre l'operazione di inserimento di una coppia chiave-valore $(k, v)$ con $T[k] = v$.
Con $T.keys$ e $T.values$ si indicano rispettivamente l'insieme delle chiavi e l'insieme dei valori contenuti in $T$.

La copertura $Q$ è quindi data da un dizionario $T$ di dimensione $|V|$, in cui le coppie chiave-valore consistono di:
\begin{itemize}
    \item Chiave: nodo $v \in V$
    \item Valore: l'insieme di insiemi componente $C \in Q$ tali per cui $v \in C$
\end{itemize}
\newpage

Nel corso dell'algoritmo un altro dizionario $T'$ viene usato allo scopo di mappare la biiezione tra gli elementi
di $\mathcal{D}(Q)$ (quindi le classi di equivalenza di $V/R_Q$) e i relativi supernodi di $G'$.
Aggiungendo coppie a $T'$, naturalmente, al termine dell'algoritmo il dizionario $T'$ dovrà raggiungere la dimensione
$|\mathcal{D}(Q)|$.

Nel ciclo a riga 3, si assegna ogni nodo del grafo decontraibile in input $G$ ad un supernodo, che esso sia
creato contestualmente o che sia il supernodo già creato corrispondente all'insieme di insiemi componente in $v$.
Essendo che le precondizioni impongono che $T$ rappresenti una copertura di $V$, l'insieme $G.V$ a riga 3
potrebbe essere sostituito con $T.keys$.
Nel ciclo a riga 15, si assegnano gli archi di $G$ ai superarchi di $G'$. Se l'arco $(v, w)$ è interno ad un
supernodo, esso viene assegnato al suo insieme di archi. Se invece l'arco collega due supernodi distinti, esso
viene assegnato al superarco corrispondente, che potrebbe essere contestualmente creato.

\begin{algorithm}[H] \floatname{algorithm}{Algoritmo}
    \caption{MAKE-DECONTRACTIBLE-GRAPH($T,G$)}\label{alg:make-decontractible-graph}
    \begin{algorithmic}[1]
        \State Sia $G' = (\mathfrak{V}, \mathfrak{E})$ un nuovo grafo decontraibile, con $\mathfrak{V} = \emptyset$
                e $\mathfrak{E} = \emptyset$
        \State Sia $T\mathcal{'}$ una nuova tabella, con insiemi di nodi come chiavi e supernodi come valori
        \For{$v\in G.V$}
            \If {$T[v]\notin T\mathcal{'}.keys$}
                \State Sia $\beta$ un nuovo supernodo con $G_{\beta}$ = $(\emptyset, \emptyset)$
                \State $\mathfrak{V}$ $\coloneqq$ $\mathfrak{V}$ $\cup$ $\{\beta\}$
                \State $T'[T[v]] \coloneqq \beta$
                \State $\alpha \coloneqq \beta$
            \Else
                \State $\alpha \coloneqq T'[T[v]]$
            \EndIf
            \State $V_{\alpha} \coloneqq V_{\alpha}$ $\cup$ $\{v\}$
            \State $v.supernode \coloneqq \alpha$
        \EndFor
        \For {$(v,w)$ in $G.E$}
            \State $\alpha \coloneqq v.supernode$
            \State $\beta \coloneqq w.supernode$
            \If {$(\alpha == \beta )$}
                \State $E_{\alpha} \coloneqq E_{\alpha} \cup \{(v,w)\}$
            \Else
                \If {$(\alpha , \beta)$ $\notin$ $\mathfrak{E}$}
                    \State $\mathfrak{E}$ $\coloneqq$ $\mathfrak{E}$ $\cup$ $\{(\alpha , \beta)\}$
                \EndIf
                \State $(\alpha , \beta).dec$ $\coloneqq$ $(\alpha , \beta).dec$ $\cup$ $\{(v, w)\}$
            \EndIf
        \EndFor
        \State \Return $G'$
    \end{algorithmic}
\end{algorithm}

In Figura~\ref{fig:make_decontractible_graph_example} sono rappresentati il grafo $G$, suddiviso in due insiemi
componente, e il corrispondente dizionario $T$ che ne rappresenta la copertura. Il dizionario $T\mathcal{'}$ \`e
rappresentato nello stato seguente all'applicazione dell'algoritmo MAKE-DECONTRACTIBLE-GRAPH con input $T$ e $G$,
contenendo come insieme di valori i supernodi ottenuti dalla contrazione di $G$.
Il risultato di MAKE-DECONTRACTIBLE-GRAPH($T$, $G$) \`e in questo caso un grafo decontraibile
$G' = (\mathfrak{V}, \mathfrak{E}, dec_{\mathfrak{V}}, dec_{\mathfrak{E}})$ con
$\mathfrak{V} = (\{\alpha_1, \alpha_2, \alpha_3\}$ e $\mathfrak{E} = \{(\alpha_1, \alpha_2), (\alpha_2, \alpha_3)\})$.

\begin{figure}[H]
    \centering
    \begin{multicols}{3}
    \hspace{-1cm}
    \begin{tabularx}{0.8\linewidth} {
        | >{\raggedright\arraybackslash}X
        | >{\centering\arraybackslash}X | }
        \multicolumn{2}{>{\hsize=\dimexpr2\hsize+2\tabcolsep+\arrayrulewidth\relax}X}{\center$T$} \\
        \hline
        $v_1$ & $\{C_1\}$\\
        \hline
        $v_2$ & $\{C_1\}$\\
        \hline
        $v_3$ & $\{C_1, C_2\}$\\
        \hline
        $v_4$ & $\{C_2\}$\\
        \hline
        $v_5$ & $\{C_2\}$\\
        \hline
    \end{tabularx}
    \columnbreak \hspace{-1.75cm}
    \begin{tabularx}{0.8\linewidth} {
        | >{\raggedright\arraybackslash}X
        | >{\centering\arraybackslash}X | }
        \multicolumn{2}{>{\hsize=\dimexpr2\hsize+2\tabcolsep+\arrayrulewidth\relax}X}{\center$T'$} \\
        \hline
        $\{C_1\}$ & $\alpha_1$\\
        \hline
        $\{C_1, C_2\}$ & $\alpha_2$\\
        \hline
        $\{C_2\}$ & $\alpha_3$\\
        \hline
    \end{tabularx}
    \columnbreak \hspace{-1cm}
    \begin{tikzpicture}

        \node[circle, draw] (A) {$v_1$};
        \node[circle, draw] (B) [above of=A, node distance=1.4cm] {$v_2$};
        \node[circle, draw] (C) [right=0cm and 1cm of B] {$v_3$};
        \node[circle, draw] (D) [right=0cm and 1cm of C] {$v_4$};
        \node[circle, draw] (E) [below of=D, node distance=1.4cm] {$v_5$};
        \begin{pgfonlayer}{background}
            \node[blue,
            circle,
            draw,
            fit=(A)(B)(C),
            label={[align=center, black]above:{$C_1$}}] (set1) {};
        \node[red,
            circle,
            draw,
            fit=(C)(D)(E),
            label={[align=center, black]above:{$C_2$}}] (set2) {};

        \end{pgfonlayer}
        \draw[myarrow] (A) -- (B);
        \draw[myarrow] (B) -- (C);
        \draw[myarrow] (A) -- (C);
        \draw[myarrow] (C) -- (D);
        \draw[myarrow] (D) -- (E);
        \draw[myarrow] (C) -- (E);
    \end{tikzpicture}
\end{multicols}
    \vspace{-30pt}
    \caption{Esempio di tabelle $T$ e $T'$ e del corrispondente grafo decontraibile}
    \label{fig:make_decontractible_graph_example}
\end{figure}

\paragraph{Complessità}
Facendo uso di tabelle di hash per rappresentare i dizionari $T$ e $T'$, l'algoritmo mantiene una complessit\`a
temporale di $\mathbf{O(|V| + |E|)}$.
Si noti infatti che:
\begin{itemize}
    \item Il primo ciclo for scorre tutti i nodi in $V$.
    Esso inizializza il dizionario $T'$, definisce i supernodi di $G'$ e assegna loro i nodi in $V$.
    La complessità del ciclo è $O(|V|)$, infatti il controllo a riga $4$ ha complessità costante,
    così come le ricerca in $T$ e $T'$ alle righe $7$ e $10$.
    \item Il secondo ciclo for scorre tutti gli archi in $E$.
    Esso assegna gli archi di $G$ ai grafi dei supernodi di $G'$ o ai superarchi di $G'$ a seconda dei supernodi
    di appartenenza dei nodi che li compongono, ottenibili in tempo $O(1)$ in quanto precedentemente calcolati nel
    primo ciclo for. La complessit\`a del ciclo \`e $O(|E|)$.
\end{itemize}

\subsection{Algoritmi di contrazione}\label{sec:algoritmi-di-contrazione}

In questa sezione illustriamo brevemente come a partire dagli algoritmi di enumerazione di sottoinsiemi
e dall'algoritmo per la contrazione costruita da una rappresentazione
disgiunta, \`e quindi possibile costruire delle procedure complete che rappresentino delle funzioni di contrazione.

\nlparagraph{Procedure per la gestione dei dizionari}\label{subsec:procedure-per-la-gestione-deli-dizionari}

Come si pu\`o notare dallo pseudo-codice dagli algoritmi di enumerazione descritti nella
Sezione~\ref{sec:algoritmi-di-enumerazione}, ciascuno di essi provvede a fornire in output una sequenza di
sottoinsiemi di nodi che rispettino il pattern strutturale individuato.
Allo scopo di costruire una funzione di contrazione, \`e però necessario che tali insiemi siano usati per definire
una copertura di nodi rappresentata attraverso un dizionario $T$, come descritto nella sezione precedente.
Per questo motivo si propongono una serie di procedure che permettano di inizializzare, aggiungere e rimuovere
sottoinsiemi da un dizionario $T$, affinché queste possano essere usate come integrazione per fornire un
adattamento dello pseudo-codice originale degli algoritmi di enumerazione presentati.
\newpage

In particolare, come riga iniziale di ciascuno di questi algoritmi, può essere aggiunta la seguente procedura per
l'inizializzazione del dizionario $T$.

\begin{algorithm}[H] \floatname{algorithm}{Algoritmo}
    \caption{INITIALIZE-TABLE($V$)}\label{alg:init-table}
    \begin{algorithmic}[1]
        \State Sia $T$ una nuova tabella
        \For{$v \in V$}
            \State $T[v] \coloneqq \emptyset$
        \EndFor
        \State \Return $T$
    \end{algorithmic}
\end{algorithm}

L'operazione di fornire in output un insieme componente potrebbe essere invece sostituita in ciascun algoritmo
con la seguente procedura per l'aggiunta di un insieme $C$ agli insiemi componente tracciati da $T$. \newline

\begin{algorithm}[H]
    \caption{ADD-COMPONENT-SET($T$, $C$)}\label{alg:add-subset}
    \begin{algorithmic}[1]
        \For{$v\in C$}
            \State $T[v]:=T[v] \cup \{C\}$
            \State $T.modified:=T.modified \cup \{v\}$
        \EndFor
    \end{algorithmic}
\end{algorithm}

Si fornisce anche la procedura per la rimozione di un insieme $C$ da $T$.

\begin{algorithm}[H] \floatname{algorithm}{Algoritmo}
    \caption{REMOVE-COMPONENT-SET($T$, $C$)}\label{alg:rem-subset}
    \begin{algorithmic}[1]
        \For{$v\in C$}
            \State $T[v]:=T[v] \setminus \{C\}$
            \State $T.modified:=T.modified \cup \{v\}$
        \EndFor
    \end{algorithmic}
\end{algorithm}

Nel caso in cui si avesse a che fare con procedure di enumerazione che forniscono in output insiemi di nodi che non
necessariamente forniscono una copertura di tutti i nodi del grafo (come nel caso dell'algoritmo di enumerazione
dei circuiti semplici), la seguente procedura può essere utilizzata al termine dell'enumerazione per aggiungere
ciascun nodo rimasto escluso dalla copertura a un insieme componente singoletto. Come evidente dallo
pseudocodice, tale procedura manterrebbe una complessit\`a temporale di $O(|V|)$, con $|V|$ il numero di nodi del grafo
decontraibile da contrarre.

\begin{algorithm}[H] \floatname{algorithm}{Algoritmo}
    \caption{ADD-SINGLETONS($T$)}\label{alg:add-singletons}
    \begin{algorithmic}[1]
        \For{$v \in T.keys$}
            \If{$T[v] == \emptyset$}
                \State $T[v] := \{v\}$
            \EndIf
        \EndFor
    \end{algorithmic}
\end{algorithm}

%Si pu\`o notare che le procedure di aggiunta e rimozione di un sottoinsieme $C$ da $T$ prevedono il tracciamento
%dei particolari nodi corrispondenti a record di $T$ modificati, attraverso l'utilizzo di un insieme associato
%alla tabella e indicato come $T_{modified}$.
%Sebbene non sia necessario per le procedure di contrazione, questo risulterà utile per rendere le procedure
%applicabili anche nelle procedure di aggiornamento dinamico al grafo contratto, come verrà trattato nel capitolo~\ref{}.
%L'insieme $T_{modified}$ permetterà, infatti, di evitare di scorrere l'intero dizionario $T$ nel momento in cui il
%grafo contratto già costruito dovrà essere aggiornato in accordo alle modifiche agli insiemi componente tracciati
%da $T$.

\newpage

\nlparagraph{Aggiunta di un sottoinsieme massimale}

Sebbene alcuni algoritmi di enumerazione possano già fornire in output sottoinsiemi di nodi tra loro massimali,
come nel caso dell'algoritmo di enumerazione delle componenti fortemente connesse e delle cricche massimali,
in generale non è garantito che ciò avvenga, come nel caso dell'algoritmo di enumerazione dei circuiti semplici.
Qualora sia necessario adattare una funzione di contrazione affinché questa sia basata sui soli insiemi di nodi
massimali forniti da un algoritmo di enumerazione, la procedura descritta di seguito può essere utilizzata in
sostituzione della normale procedura ADD-COMPONENT-SET\@.
Infatti, questa procedura permette controllare contestualmente alla sua aggiunta a $T$ se un dato sottoinsieme $C$ di
nodi sia massimale rispetto agli altri insiemi già tracciati in $T$ e, se si, individuare quali eventuali sottoinsiemi
in $T$ siano inclusi in $C$, affinché possano essere rimossi.

\begin{algorithm}[H]
    \caption{ADD-MAXIMAL-SUBSET($T$, $C$)}\label{alg:add-maximal-subset}
    \begin{algorithmic}[1]
        \If{$\bigcap_{v \in C}T[v] == \emptyset$}
            \State $Q :=$ FIND-SUBSETS($T$, $C$)
            \For {$S \in Q$}
                \State REMOVE-SUBSET($T$, $S$)
            \EndFor
            \State ADD-SUBSET($T$, $C$)
        \EndIf
    \end{algorithmic}
\end{algorithm}
\begin{algorithm}[H] \floatname{algorithm}{Algoritmo}
    \caption{FIND-SUBSETS($T$, $C$)}\label{alg:find-subsets}
    \begin{algorithmic}[1]
        \State Sia $T_c$ una nuova tabella per contare le occorrenze dei sottoinsiemi
        \For{$v \in C$}
            \For{$S \in T[v]$}
                \State $T_c[S] := T_c[S] + 1$
            \EndFor
        \EndFor
        \State $Q := \emptyset$
        \For {$S \in T_c$}
            \If{$T_c[S] == |C|$}
                \State $Q := Q \cup \{S\}$
            \EndIf
        \EndFor
        \State \Return $Q$
    \end{algorithmic}
\end{algorithm}

In particolare, l'algoritmo ADD-MAXIMAL-COMPONENT-SET provvede innanzitutto a verificare a riga 1 che l'insieme $C$ non
sia gi\`a incluso in un altro insieme presente tra gli insiemi di $T$, ovvero che non sia certamente
massimale.
Questo viene verificato controllando se tutti i nodi in $C$ condividano l'appartenenza ad uno o pi\`u insiemi gi\`a
presenti. \newline
Se l'insieme $C$ \`e massimale rispetto ai nodi trovati, successivamente l'algoritmo provvede ad individuare
gli eventuali insiemi presenti in $T$ che siano sottoinsiemi di $C$ attraverso la procedura FIND-SUBSETS\@.
Essa conteggia il numero di volte che ciascun sottoinsieme appare tra i nodi di $C$ rispetto al dizionario $T$.
Un conteggio pari alla cardinalit\`a dell'insieme stesso testimonia che tutti i suoi nodi sono presenti in $C$.

\paragraph{Complessit\`a}
Con l'utilizzo di insiemi basati su hash, in cui la complessit\`a per verificare l'appartenenza ad un insieme \`e
$O(1)$, la complessit\`a della procedura di ADD-MAXIMAL-COMPONENT-SET risulta essere $\mathbf{O(|C| c_{max}})$ nel caso peggiore,
dove $|C|$ \`e la cardinalit\`a dell'insieme $C$ in input e $c_{max}$ \`e il limite superiore al numero di sottoinsiemi
massimali già tracciati dalla tabella $T$, che risulta essere massimo quando la cardinalit\`a di questi insiemi
\`e pari alla met\`a del numero di nodi del grafo $G$ che si sta contraendo.
Si noti che la complessit\`a apportata da $c_{max}$ non pu\`o mai essere
superiore all' esponenziale, in quanto, in generale, il numero massimo di insiemi massimali che possono essere
individuati in un insieme di $n$ elementi rimarr\`a sempre inferiore al numero di tutti i possibili sottoinsiemi,
che è $2^n$.
Sebbene il termine $c_{max}$ possa essere considerato come un $O(2^n)$ nel calcolo della complessit\`a (con $n$
il numero di nodi del grafo $G$), esso pu\`o essere calcolato in modo pi\`u preciso come segue:

\begin{equation*}
    c_{max} \quad = \quad \binom{n}{\lfloor \frac{n}{2} \rfloor} = \frac{n\cdot(n-1)\cdot\ldots\cdot(n/2)}{\frac{n}{2}!}
\end{equation*}

La complessit\`a della procedura pu\`o essere calcolata notando che l'intersezione nel controllo a riga 1 pu\`o
impiegare al pi\`u $O((|C|-1) \cdot max_{v \in C}\{|T[v]|\})$.
Tuttavia, eseguendo le intersezioni a partire dall'insieme $T[v]$ pi\`u piccolo, individuato in tempo lineare a
$|C|$, la complessit\`a pu\`o essere ridotta a $O((|C|-1) \cdot min_{v \in C}\{|T[v]|\})$.
In ogni caso, questa complessit\`a pu\`o essere riscritta come $O(|C| \cdot c_{max})$, considerando il caso peggiore
in cui tutti i nodi in $C$ appartengono agli insiemi costruiti tramite unione tra i possibili sottoinsiemi massimali
tra i restanti $|V|-|C|$ nodi del grafo e $C$, tutti di egual cardinalit\`a.
Inoltre la procedura FIND-SUBSETS impiega anch'essa al pi\`u $O(|C| \cdot c_{max})$ tempo,
in quanto il ciclo for a riga 2 scorre tutti i nodi in $C$, mentre il ciclo for a riga 3 scorre tutti i
sottoinsiemi massimali presenti in $T$ per ogni nodo in $C$, che sono al pi\`u pari a tutti i sottoinsiemi massimali
ottenibili a partire dai restanti $|V|-1$ nodi del grafo.
Inoltre il ciclo for a riga 8 scorre tutti i sottoinsiemi massimali trovati, che sono al pi\`u pari a $c_{max} - 1$,
considerato che l'insieme $C$ deve essere anch'esso massimale.

\nlparagraph{Procedure complete di contrazione}

Forniamo ora una descrizione della struttura di un algoritmo generico che rappresenti la procedura completa di
contrazione di un grafo decontraibile $G$, ovvero che permetta di realizzare una funzione di contrazione $f_C$
che possa essere usata nella definizione di un grafo multi-livello.

Esso pu\`o essere costituito dalle seguenti fasi:
\begin{enumerate}
    \item \textbf{Fase di pre-elaborazione} (opzionale) \\
    Dato il grafo iniziale $G$, si produce un nuovo grafo $H$ adatto ad essere fornito in input all'algoritmo
    della fase successiva.
    Questo potrebbe essere utile allo scopo di adattare il grafo all'algoritmo, migliorare le performance o
    costruire un nuovo grafo che rappresenti particolari relazioni tra i nodi del grafo originale oltre alla
    semplice adiacenza.
    \item \textbf{Fase di enumerazione dei sottoinsiemi} \\
    Nel grafo $H$ vengono riconosciuti i pattern strutturali dallo specifico algoritmo di enumerazione e tracciati
    sotto forma di insiemi componente composti dai nodi che fanno parte di tali pattern.
    Il risultato \`e la mappatura $T$ tra i nodi del grafo $H$ e l'insieme dei sottoinsiemi di cui fanno parte.
    \item \textbf{Fase di contrazione} \\
    Viene restituita la contrazione di $G$ costruita sulla riduzione disgiunta dei sottoinsiemi individuati,
    secondo la logica dell'algoritmo MAKE-CONTRACTIBLE-GRAPH\@.
\end{enumerate}

\begin{algorithm}[H]
    \caption{GENERIC-CONTRACTION($G$)}\label{alg:generic-contraction}
    \begin{algorithmic}[1]
        \State Let $H$ be a graph obtained from $G$ by a preprocessing procedure
        \State Let $T$ be a table obtained from $H$ by a generic subset enumerating algorithm
        \State $J$ = MAKE-CONTRACTIBLE-GRAPH($T$, $G$)
        \State return $J$
    \end{algorithmic}
\end{algorithm}

Gli algoritmi di contrazione relativi ai problemi di enumerazione trattati potrebbero seguire la struttura sopra
descritta nei seguenti modi:
\begin{itemize}
    \item Per la contrazione per componenti fortemente connesse, la fase di preprocessing potrebbe essere
    essere saltata, in quanto l'algoritmo di Kosaraju-Sharir e la sua versione adattata non necessitano di alcuna
    modifica al grafo orientato in input.
    \item Per la contrazione per cricche può essere utilizzata una fase di preprocessing per la costruzione
    della relativa versione non orientata di un grafo, che potrebbe sostituire tutti gli archi orientati con
    archi non orientati, oppure potrebbe mantenere un arco non orientato solo per quelle coppie di nodi
    mutualmente adiacenti.
    La fase di enumerazione può essere realizzata dalla versione adattata dell'algoritmo di Bron-Kerbosch.
    \item Per la contrazione per circuiti semplici pu\`o essere utilizzata una fase di preprocessing per
    l'eliminazione di tutti gli archi tra nodi che non facciano parte delle stesse componenti fortemente connesse
    allo scopo di migliorare le performance della fase di enumerazione, che può essere realizzata dall'algoritmo
    di Johnson nella sua versione adattata.
\end{itemize}