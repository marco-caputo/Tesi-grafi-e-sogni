\chapter{Grafi multilevello per la rappresentazione di sogni}

Nel contesto psichiatrico, la rappresentazione di testi sui sogni attraverso grafi si \`e rivelato un'utile
strumento a supporto della diagnosi di disturbi come la schizofrenia o il disturbo bipolare
~\cite{mota2014graph}, nonch\'e la possibilit\'a di predire l'insorgenza della schizofrenia con un anticipo di
sei mesi rispetto alla diagnosi clinica~\cite{mota2017thought}.

Tali studi evidenziando di come l'attenzione ad aspetti sintattici della descrizione di sogni possano fornire
informazioni significative sullo stato mentale di un individuo rispetto a quando le stesse tecniche sono applicate
ad altri tipi di testi prodotti dallo stesso.

Aspetti di rilievo del cos\'i detto \textit{speec graph}, ovvero il grafo ottenuto dal report orale di un sogno in cui
ogni parola \`e rappresentata come un nodo e ogni connessione temporale tra parole consecutive come un arco diretto,
sono stati la frequenza di utilizzo delle parole, la presenza di cicli su parole ricorrenti, la connettivit\'a delle
parole e la loro organizzazione in componenti e componenti fortemente connesse. \newline





